Funkcja $f_b$ zapisana w postaci maxtermów wygląda następująco.
    \[f_b(x_3, x_2, x_1, x_0) = \prod (5, 6, (10, 11, 12, 13, 14, 15))\]
    \begin{figure}[h]
        \centering
        \begin{karnaugh-map}[4][4][1][$x_1x_0$][$x_3x_2$]
            \minterms{0, 1, 2, 3, 4, 7, 8, 9}
            \maxterms{5, 6}
            \indeterminants{10, 11, 12, 13, 14, 15}
            \implicant{5}{13}
            \implicant{6}{14}
        \end{karnaugh-map}
        \caption{Tabela dla funkcji \textrm{b}}
        \label{fig:fb}
    \end{figure}
    Minimalizację funkcji $f_b(x_3, x_2, x_1, x_0)$ za pomocą tablicy Karnaugha przedstawiono na Rys.~\ref{fig:fb}.
    Znaleziono 2 implikanty proste, z których wszystkie były niezbędne (\textrm{EIP}).
    \begin{equation}
        \label{eq:fb}
        f_b(x_3, x_2, x_1, x_0) = (x_0' + x_1 + x_2')(x_0 + x_1' + x_2')
    \end{equation}
    Równanie~\ref{eq:fb} jest zminimalizowanym równaniem funkcji  $f_b$.
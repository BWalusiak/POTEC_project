Funkcja $f_b$ zapisana w postaci maxtermów wygląda następująco.
\[f_b(x_3, x_2, x_1, x_0) = \prod (0, 1, 2, 3, 4, 7, 8, 9, (10, 11, 12, 13, 14, 15))\]
\begin{figure}[h]
    \centering
    \begin{karnaugh-map}[4][4][1][$x_1x_0$][$x_3x_2$]
        \minterms{5, 6}
        \maxterms{0, 1, 2, 3, 4, 7, 8, 9}
        \indeterminants{10, 11, 12, 13, 14, 15}
        \implicant{0}{2}
        \implicant{8}{10}
        \implicant{0}{8}
        \implicant{3}{11}
    \end{karnaugh-map}
    \caption{Tabela dla funkcji \textrm{b}}
    \label{fig:fb}
\end{figure}
Minimalizację funkcji $f_b(x_3, x_2, x_1, x_0)$ za pomocą tablicy Karnaugha przedstawiono na Rys.~\ref{fig:fb}.
Znaleziono 4 implikanty proste, z których wszystkie były niezbędne (\textrm{EIP}).
\begin{equation}
    \label{eq:fb}
    f_b(x_3, x_2, x_1, x_0) = (x_2+x_0)x_1'(x_2'+x_0')x_3'
\end{equation}
Równanie~\ref{eq:fb} jest zminimalizowanym równaniem funkcji  $f_b$.

%! Author = Bartosz Walusiak
%! Date = 11/16/2019

% Preamble
\documentclass[a4paper, 12pt]{article}

\usepackage{float}
\usepackage[T1]{fontenc}
\usepackage{lmodern}
\usepackage[polish]{babel}
\usepackage[utf8]{inputenc}
\usepackage{polski}
\usepackage{hyperref}
\usepackage{karnaugh-map}
\usepackage{amsmath}
\usepackage{ulem}
\usepackage{bm}

\newcounter{implicant_counter}
\newcommand{\imp}{
I\textsubscript{\arabic{implicant_counter}}
\label{imp:\arabic{implicant_counter}}
\refstepcounter{implicant_counter}
}

\begin{document}
    \title{Projekt nr 1 POTEC}
    \author{Monika Lewandowska, Bartosz Walusiak, Krzysztof Zdulski}
    \date{\today}
    \maketitle

    \tableofcontents

    \newpage
    \section{Zadanie 1 - Minimalizacja funkcji boolowskiej za pomocą tablic Karnaugha dla mintermów}\label{sec:task-1}
    \subsection{Funkcja g}\label{subsec:fun-g}
    Funkcja $f_g$ zapisana w postaci mintermów wygląda następująco.
\[f_g(x_3, x_2, x_1, x_0) = \sum (2, 3, 4, 5, 6, 8, 9, (10, 11, 12, 13, 14, 15))\]
\begin{figure}[H]
    \centering
    \begin{karnaugh-map}[4][4][1][$x_1x_0$][$x_3x_2$]
        \minterms{2, 3, 4, 5, 6, 8, 9}
        \maxterms{0, 1, 7}
        \indeterminants{10, 11, 12, 13, 14, 15}
        \implicant{2}{10}
        \implicant{12}{10}
        \implicant{4}{13}
        \implicantedge{3}{2}{11}{10}
    \end{karnaugh-map}
    \caption{Tablica dla funkcji \textrm{g}}
    \label{fig:fg}
\end{figure}
Minimalizację funkcji $f_g(x_3, x_2, x_1, x_0)$ za pomocą tablicy Karnaugha przedstawiono na Rys.~\ref{fig:fg}.
Znaleziono 4 implikanty proste, z których wszystkie były niezbędne (\textrm{EIP}).
\begin{equation}
    \label{eq:fg}
    f_g(x_3, x_2, x_1, x_0) = x_3 + x_0'x_1 + x_1x_2' + x_1'x_2
\end{equation}
Równanie~\ref{eq:fg} jest zminimalizowanym równaniem funkcji  $f_g$.

    \newpage
    \section{Zadanie 2 - Minimalizacja funkcji boolowskiej za pomocą tablic Karnaugha dla maxtermów}\label{sec:task-2}
    \subsection{Funkcja a}\label{subsec:fun-a}
    Funkcja $f_a$ zapisana w postaci makstermów wygląda następująco:
\[f_a(x_3, x_2, x_1, x_0) = \prod (0, 2, 3, 5, 6, 7, 8, 9, (10, 11, 12, 13, 14, 15))\]
\begin{figure}[h]
    \centering
    \begin{karnaugh-map}[4][4][1][$x_{1}x_0$][$x_{3}x_2$]
        \minterms{1,4}
        \maxterms{0, 2, 3, 5, 6, 7, 8, 9}
        \indeterminants{10, 11, 12, 13, 14, 15}
        \implicantcorner
        \implicant{12}{10}
        \implicant{3}{10}
        \implicant{5}{15}
        \end{karnaugh-map}
    \caption{Tablica dla funkcji \textrm{a}}
    \label{fig:fa}
\end{figure}
Minimalizację funkcji $f_a(x_3, x_2, x_1, x_0)$ za pomocą tablicy Karnaugha przedstawiono na Rys.~\ref{fig:fa}.
Znaleziono 4 implikanty proste, z których wszystkie są niezbędne (\textrm{EIP}).
\begin{equation}
    \label{eq:fa}
    f_a(x_3, x_2, x_1, x_0) =  x_3' x_1'  (x_2 + x_0) (x_2'+x_0')
\end{equation}
Równanie~\ref{eq:fa} jest zminimalizowanym równaniem funkcji $f_a$.


    \newpage
    \subsection{Funkcja b}\label{subsec:fun-b}
    Funkcja $f_b$ zapisana w postaci maxtermów wygląda następująco.
\[f_b(x_3, x_2, x_1, x_0) = \prod (5, 6, (10, 11, 12, 13, 14, 15))\]
\begin{figure}[h]
    \centering
    \begin{karnaugh-map}[4][4][1][$x_1x_0$][$x_3x_2$]
        \minterms{0, 1, 2, 3, 4, 7, 8, 9}
        \maxterms{5, 6}
        \indeterminants{10, 11, 12, 13, 14, 15}
        \implicant{5}{13}
        \implicant{6}{14}
    \end{karnaugh-map}
    \caption{Tabela dla funkcji \textrm{b}}
    \label{fig:fb}
\end{figure}
Minimalizację funkcji $f_b(x_3, x_2, x_1, x_0)$ za pomocą tablicy Karnaugha przedstawiono na Rys.~\ref{fig:fb}.
Znaleziono 2 implikanty proste, z których wszystkie były niezbędne (\textrm{EIP}).
\begin{equation}
    \label{eq:fb}
    f_b(x_3, x_2, x_1, x_0) = (x_0' + x_1 + x_2')(x_0 + x_1' + x_2')
\end{equation}
Równanie~\ref{eq:fb} jest zminimalizowanym równaniem funkcji  $f_b$.

    \newpage
    \section{Zadanie 3 - Minimalizacja funkcji boolowskiej z wykorzystaniem systematycznej metody ekspansji}\label{sec:task-3}
    \subsection{Funkcja c}\label{subsec:fun-c}
    %! Suppress = MissingImport
\setcounter{implicant_counter}{0}

Funkcja $f_c$ rozpisana na macierze $F$ i $R$ ma postać:
\begin{center}
    \begin{tabular}[t]{ |c|c c c c|}
        \hline
        $F$ & $x_3$ & $x_2$ & $x_1$ & $x_0$ \\
        \hline
        $k_0$ & 0 & 0 & 0 & 0 \\
        $k_1$ & 0 & 0 & 0 & 1 \\
        $k_2$ & 0 & 0 & 1 & 1 \\
        $k_3$ & 0 & 1 & 0 & 0 \\
        $k_4$ & 0 & 1 & 0 & 1 \\
        $k_5$ & 0 & 1 & 1 & 0 \\
        $k_6$ & 0 & 1 & 1 & 1 \\
        $k_7$ & 1 & 0 & 0 & 0 \\
        $k_8$ & 1 & 0 & 0 & 1 \\
        \hline
    \end{tabular}
    \hspace{1cm}
    \begin{tabular}[t]{ |c|c c c c| }
        \hline
        $R$ & $x_3$ & $x_2$ & $x_1$ & $x_0$ \\
        \hline
        & 0 & 0 & 1 & 0 \\
        \hline
    \end{tabular}
\end{center}

Rozpoczynamy od wyznaczenia wszystkich implikantów prostych korzystając z macierzy blokujących dla kolejnych kostek
$k_0-k_8$.

\begin{table}[H]
    \centering
    \begin{tabular}[t]{ |c|c c c c| }
        \hline
        $k_0$ & 0 & 0 & 0 & 0 \\
        \hline\hline
        $B_0$ & $x_3$ & $x_2$ & $x_1$ & $x_0$ \\
        \hline
        & 0 & 0 & \textbf{1} & 0 \\
        \hline
    \end{tabular}
    \caption{Macierz blokująca $B_0$} \label{tab:b0}
\end{table}

Dla macierzy blokującej przedstawionej w Tablicy~\ref{tab:b0} znajdujemy jedno minimalne pokrycie kolumnowe
$\bm{L'=\{1\}}$ i wynikający z niego implikant prosty $\imp=({*}{*}0{*})=x_1'$.

\begin{table}[H]
    \centering
    \begin{tabular}[t]{ |c|c c c c| }
        \hline
        $k_1$ & 0 & 0 & 0 & 1 \\
        \hline\hline
        $B_1$ & $x_3$ & $x_2$ & $x_1$ & $x_0$ \\
        \hline
        & 0 & 0 & \textbf{1} & \textbf{1} \\
        \hline
    \end{tabular}
    \caption{Macierz blokująca $B_1$} \label{tab:b1}
\end{table}

Dla macierzy blokującej przedstawionej w Tablicy~\ref{tab:b1} znajdujemy dwa minimalne pokrycia kolumnowe
$\bm{L'=\{0\}}$, $L'=\{1\}$ i
wynikający z nich implikanty proste $\imp=({*}{*}{*}1)=x_0$, $\imp=({*}{*}0{*})=x_1'$.

\begin{table}[H]
    \centering
    \begin{tabular}[t]{ |c|c c c c| }
        \hline
        $k_2$ & 0 & 0 & 1 & 1 \\
        \hline\hline
        $B_2$ & $x_3$ & $x_2$ & $x_1$ & $x_0$ \\
        \hline
        & 0 & 0 & 0 & \textbf{1} \\
        \hline
    \end{tabular}
    \caption{Macierz blokująca $B_2$} \label{tab:b2}
\end{table}

Dla macierzy blokującej przedstawionej w Tablicy~\ref{tab:b0} znajdujemy jedno minimalne pokrycie kolumnowe
$\bm{L'=\{0\}}$ i wynikający z niego implikant prosty $\imp=({*}{*}{*}1)=x_0$.

\begin{table}[H]
    \centering
    \begin{tabular}[t]{ |c|c c c c| }
        \hline
        $k_3$ & 0 & 1 & 0 & 0 \\
        \hline\hline
        $B_3$ & $x_3$ & $x_2$ & $x_1$ & $x_0$ \\
        \hline
        & 0 & \textbf{1} & \textbf{1} & 0 \\
        \hline
    \end{tabular}
    \caption{Macierz blokująca $B_3$} \label{tab:b3}
\end{table}

Dla macierzy blokującej przedstawionej w Tablicy~\ref{tab:b1} znajdujemy dwa minimalne pokrycia kolumnowe
$\bm{L'=\{1\}}$, $L'=\{2\}$ i
wynikający z nich implikanty proste $\imp=({*}{*}0{*})=x_1'$, $\imp=({*}1{*}{*})=x_2$.

\begin{table}[H]
    \centering
    \begin{tabular}[t]{ |c|c c c c| }
        \hline
        $k_4$ & 0 & 1 & 0 & 1 \\
        \hline\hline
        $B_4$ & $x_3$ & $x_2$ & $x_1$ & $x_0$ \\
        \hline
        & 0 & \textbf{1} & \textbf{1} & \textbf{1} \\
        \hline
    \end{tabular}
    \caption{Macierz blokująca $B_4$} \label{tab:b4}
\end{table}

Dla macierzy blokującej przedstawionej w Tablicy~\ref{tab:b1} znajdujemy trzy minimalne pokrycia kolumnowe
$\bm{L'=\{0\}}$, $L'=\{1\}$, $L'=\{2\}$ i
wynikający z nich implikanty proste $\imp=({*}{*}{*}1)=x_1$, $\imp=({*}{*}0{*})=x_1'$, $\imp=({*}1{*}{*})=x_2$.

\begin{table}[H]
    \centering
    \begin{tabular}[t]{ |c|c c c c| }
        \hline
        $k_5$ & 0 & 1 & 1 & 0 \\
        \hline\hline
        $B_5$ & $x_3$ & $x_2$ & $x_1$ & $x_0$ \\
        \hline
        & 0 & \textbf{1} & 0 & 0 \\
        \hline
    \end{tabular}
    \caption{Macierz blokująca $B_5$} \label{tab:b5}
\end{table}

Dla macierzy blokującej przedstawionej w Tablicy~\ref{tab:b0} znajdujemy jedno minimalne pokrycie kolumnowe
$\bm{L'=\{2\}}$ i wynikający z niego implikant prosty $\imp=({*}1{*}{*})=x_2$.

\begin{table}[H]
    \centering
    \begin{tabular}[t]{ |c|c c c c| }
        \hline
        $k_6$ & 0 & 1 & 1 & 1 \\
        \hline\hline
        $B_6$ & $x_3$ & $x_2$ & $x_1$ & $x_0$ \\
        \hline
        & 0 & \textbf{1} & 0 & \textbf{1} \\
        \hline
    \end{tabular}
    \caption{Macierz blokująca $B_6$} \label{tab:b6}
\end{table}

Dla macierzy blokującej przedstawionej w Tablicy~\ref{tab:b1} znajdujemy dwa minimalne pokrycia kolumnowe
$\bm{L'=\{0\}}$, $L'=\{2\}$ i
wynikający z nich implikanty proste $\imp=({*}{*}{*}1)=x_0$, $\imp=({*}1{*}{*})=x_2$.

\begin{table}[H]
    \centering
    \begin{tabular}[t]{ |c|c c c c| }
        \hline
        $k_7$ & 1 & 0 & 0 & 0 \\
        \hline\hline
        $B_7$ & $x_3$ & $x_2$ & $x_1$ & $x_0$ \\
        \hline
        & \textbf{1} & 0 & \textbf{1} & 0 \\
        \hline
    \end{tabular}
    \caption{Macierz blokująca $B_7$} \label{tab:b7}
\end{table}

Dla macierzy blokującej przedstawionej w Tablicy~\ref{tab:b1} znajdujemy dwa minimalne pokrycia kolumnowe
$\bm{L'=\{1\}}$, $L'=\{3\}$ i
wynikający z nich implikanty proste $\imp=({*}{*}0{*})=x_1'$, $\imp=(1{*}{*}{*})=x_3$.

\begin{table}[H]
    \centering
    \begin{tabular}[t]{ |c|c c c c| }
        \hline
        $k_8$ & 1 & 0 & 0 & 1 \\
        \hline\hline
        $B_8$ & $x_3$ & $x_2$ & $x_1$ & $x_0$ \\
        \hline
        & \textbf{1} & 0 & \textbf{1} & \textbf{1} \\
        \hline
    \end{tabular}
    \caption{Macierz blokująca $B_8$} \label{tab:b8}
\end{table}

Dla macierzy blokującej przedstawionej w Tablicy~\ref{tab:b1} znajdujemy trzy minimalne pokrycia kolumnowe
$\bm{L'=\{0\}}$, $L'=\{1\}$, $L'=\{3\}$ i
wynikający z nich implikanty proste $\imp=({*}{*}{*}1)=x_1$, $\imp=({*}{*}0{*})=x_1'$, $\imp=(1{*}{*}{*})=x_3$.

\begin{table}[H]
    \centering
    \begin{tabular}[t]{ |c|c| }
        \hline
        $I_0$ & ${*}{*}0{*}$ \\
        $I_1$ & ${*}{*}{*}1$ \\
        \sout{$I_2$} & \sout{${*}{*}0{*}$} \\
        \sout{$I_3$} & \sout{${*}{*}{*}1$} \\
        \sout{$I_4$} & \sout{${*}{*}0{*}$}\\
        $I_5$ & ${*}1{*}{*}$ \\
        \sout{$I_6$} & \sout{${*}{*}{*}1$} \\
        \sout{$I_7$} & \sout{${*}{*}0{*}$} \\
        \sout{$I_8$} & \sout{${*}1{*}{*}$} \\
        $I_8$ & $1{*}{*}{*}$ \\
        \sout{$I_9$} & \sout{${*}1{*}{*}$} \\
        \sout{$I_{10}$} & \sout{${*}{*}{*}1$} \\
        \sout{$I_{11}$} &\sout{ ${*}1{*}{*}$} \\
        \sout{$I_{12}$} & \sout{${*}{*}0{*}$} \\
        \sout{ $I_{13}$} & \sout{$1{*}{*}{*}$} \\
        \sout{$I_{14}$} & \sout{${*}{*}{*}1$} \\
        \sout{$I_{15}$} & \sout{${*}{*}0{*}$} \\
        \sout{$I_{16}$} & \sout{$1{*}{*}{*}$} \\
        \hline
    \end{tabular}
    \caption{Wszystkie implikanty proste} \label{tab:all-implicants}
\end{table}
%TODO: Dodać komentarz do tabeki wszytkich implikantów

\begin{table}[H]
    \centering
    \begin{tabular}[t]{ |c||c|c|c|c| }
        \hline
        & $I_0 = {*}{*}0{*}$ & $I_1 = {*}{*}{*}1$ & $I_5 = {*}1{*}{*}$ & $I_8 = 1{*}{*}{*}$ \\
        \hline
        \hline
        $k_0 = 0000$ & \textbf{1} & 0 & 0 & 0 \\
        \hline
        \sout{$k_1 = 0001$} & \sout{1} & \sout{1} & \sout{0} & \sout{0} \\
        \hline
        $k_2 = 0011$ & 0 & \textbf{1} & 0 & 0 \\
        \hline
        \sout{$k_3 = 0100$} & \sout{1} & \sout{0} & \sout{1} & \sout{0} \\
        \hline
        \sout{$k_4 = 0101$} & \sout{1} & \sout{1} & \sout{1} & \sout{0} \\
        \hline
        $k_5 = 0110$ & 0 & 0 & \textbf{1} & 0 \\
        \hline
        \sout{$k_6 = 0111$} & \sout{0} & \sout{1} & \sout{1} & \sout{0} \\
        \hline
        \sout{$k_7 = 1000$} & \sout{1} & \sout{0} & \sout{0} & \sout{1} \\
        \hline
        \sout{$k_8 = 1001$} & \sout{1} & \sout{1} & \sout{0} & \sout{1} \\
        \hline
    \end{tabular}
    \caption{} \label{tab:min-block}
\end{table}

Minimalne pokrycie kolumnowe implikantów prostych $L' = \{I_0, I_1, I_5\}$.
$f_d(x_3, x_2, x_1, x_0) = x_1' + x_0 + x_2$

    \newpage
    \subsection{Funkcja d}\label{subsec:fun-d}
    %! Suppress = MissingImport
\setcounter{implicant_counter}{0}

Funkcja $f_d$ rozpisana na macierze $F$ i $R$ ma postać:
\begin{center}
    \begin{tabular}[t]{ |c|c c c c| }
        \hline
        $F$ & $x_3$ & $x_2$ & $x_1$ & $x_0$ \\
        \hline
        $k_0$ & 0 & 0 & 0 & 0 \\
        $k_1$ & 0 & 0 & 1 & 0 \\
        $k_2$ & 0 & 0 & 1 & 1 \\
        $k_3$ & 0 & 1 & 0 & 1 \\
        $k_4$ & 0 & 1 & 1 & 0 \\
        $k_5$ & 1 & 0 & 0 & 0 \\
        $k_6$ & 1 & 0 & 0 & 1 \\
        \hline
    \end{tabular}
    \hspace{1cm}
    \begin{tabular}[t]{ |c|c c c c| }
        \hline
        $R$ & $x_3$ & $x_2$ & $x_1$ & $x_0$ \\
        \hline
        & 0 & 0 & 0 & 1 \\
        & 0 & 1 & 0 & 0 \\
        & 0 & 1 & 1 & 1 \\
        \hline
    \end{tabular}
\end{center}

Rozpoczynamy od wyznaczenia wszystkich implikantów prostych korzystając z macierzy blokujących dla kolejnych kostek
$k_0-k_6$.
\begin{table}[H]
    \centering
    \begin{tabular}[t]{ |c|c c c c| }
        \hline
        $k_0$ & 0 & 0 & 0 & 0 \\
        \hline\hline
        $B_0$ & $x_3$ & $x_2$ & $x_1$ & $x_0$ \\
        \hline
        & 0 & 0 & 0 & \textbf{1} \\
        & 0 & \textbf{1} & 0 & 0 \\
        & \sout{0} & \sout{1} & \sout{1} & \sout{1} \\
        \hline
    \end{tabular}
    \caption{Macierz blokująca $B_0$} \label{tab:b0d}
\end{table}
Dla macierzy blokującej przedstawionej w Tablicy~\ref{tab:b0d} znajdujemy jedno minimalne pokrycie kolumnowe
$\bm{L'=\{2,0\}}$ i wynikający z niego implikant prosty $\imp{d}=({*}0{*}0)=x_2'x_0'$.

\begin{table}[H]
    \centering
    \begin{tabular}[t]{ |c|c c c c| }
        \hline
        $k_1$ & 0 & 0 & 1 & 0 \\
        \hline\hline
        $B_1$ & $x_3$ & $x_2$ & $x_1$ & $x_0$ \\
        \hline
        & 0 & 0 & \textbf{1} & \textbf{1} \\
        & 0 & 1 & \textbf{1} & 0 \\
        & 0 & 1 & 0 & \textbf{1} \\
        \hline
    \end{tabular}
    \caption{Macierz blokująca $B_1$} \label{tab:b1d}
\end{table}
Dla macierzy blokującej przedstawionej w Tablicy~\ref{tab:b1d} znajdujemy trzy minimalne pokrycia kolumnowe
$\bm{L'=\{1,0\}}$, $L'=\{2,0\}$, $L'=\{2,1\}$ i
wynikające z nich implikanty proste $\imp{d}=({*}{*}10)=x_{1}x_0'$, $\imp{d}=({*}0{*}0)=x_2'x_0'$, $\imp{d}=({*}01{*})=x_2'x_1$.

\begin{table}[H]
    \centering
    \begin{tabular}[t]{ |c|c c c c| }
        \hline
        $k_2$ & 0 & 0 & 1 & 1 \\
        \hline\hline
        $B_2$ & $x_3$ & $x_2$ & $x_1$ & $x_0$ \\
        \hline
        & 0 & 0 & \textbf{1} & 0 \\
        & \sout{0} & \sout{\textbf{1}} & \sout{\textbf{1}} & \sout{1} \\
        & 0 & \textbf{1} & 0 & 0 \\
        \hline
    \end{tabular}
    \caption{Macierz blokująca $B_2$} \label{tab:b2d}
\end{table}
Dla macierzy blokującej przedstawionej w Tablicy~\ref{tab:b2d} znajdujemy jedno minimalne pokrycie kolumnowe
$\bm{L'=\{2,1\}}$ i wynikający z niego implikant prosty $\imp{d}=({*}01{*})=x_2'x_1$.

\begin{table}[H]
    \centering
    \begin{tabular}[t]{ |c|c c c c| }
        \hline
        $k_3$ & 0 & 1 & 0 & 1 \\
        \hline\hline
        $B_3$ & $x_3$ & $x_2$ & $x_1$ & $x_0$ \\
        \hline
        & 0 & \textbf{1} & 0 & 0 \\
        & 0 & 0 & 0 & \textbf{1} \\
        & 0 & 0 & \textbf{1} & 0 \\
        \hline
    \end{tabular}
    \caption{Macierz blokująca $B_3$} \label{tab:b3d}
\end{table}
Dla macierzy blokującej przedstawionej w Tablicy~\ref{tab:b3d} znajdujemy jedno minimalne pokrycie kolumnowe
$\bm{L'=\{2,1,0\}}$ i wynikający z niego implikant prosty $\imp{d}=({*}101)=x_{2}x_1'x_0$.

\begin{table}[H]
    \centering
    \begin{tabular}[t]{ |c|c c c c| }
        \hline
        $k_4$ & 0 & 1 & 1 & 0 \\
        \hline\hline
        $B_4$ & $x_3$ & $x_2$ & $x_1$ & $x_0$ \\
        \hline
        & \sout{0} & \sout{1} & \sout{\textbf{1}} & \sout{\textbf{1}} \\
        & 0 & 0 & \textbf{1} & 0 \\
        & 0 & 0 & 0 & \textbf{1} \\
        \hline
    \end{tabular}
    \caption{Macierz blokująca $B_4$} \label{tab:b4d}
\end{table}
Dla macierzy blokującej przedstawionej w Tablicy~\ref{tab:b4d} znajdujemy jedno minimalne pokrycie kolumnowe
$\bm{L'=\{1,0\}}$ i wynikający z niego implikant prosty $\imp{d}=({*}{*}10)=x_{1}x_0'$.

\begin{table}[H]
    \centering
    \begin{tabular}[t]{ |c|c c c c| }
        \hline
        $k_5$ & 1 & 0 & 0 & 0 \\
        \hline\hline
        $B_5$ & $x_3$ & $x_2$ & $x_1$ & $x_0$ \\
        \hline
        & \textbf{1} & 0 & 0 & 1 \\
        & \textbf{1} & 1 & 0 & 0 \\
        & \sout{\textbf{1}} & \sout{1} & \sout{1} & \sout{1} \\
        \hline
    \end{tabular}
    \caption{Macierz blokująca $B_5$} \label{tab:b5d}
\end{table}
Dla macierzy blokującej przedstawionej w Tablicy~\ref{tab:b5d} znajdujemy jedno minimalne pokrycie kolumnowe
$\bm{L'=\{3\}}$ i wynikający z niego implikant prosty $\imp{d}=(1{*}{*}{*})=x_3$.

\begin{table}[H]
    \centering
    \begin{tabular}[t]{ |c|c c c c| }
        \hline
        $k_6$ & 1 & 0 & 0 & 1 \\
        \hline\hline
        $B_6$ & $x_3$ & $x_2$ & $x_1$ & $x_0$ \\
        \hline
        & \textbf{1} & 0 & 0 & 0 \\
        & \sout{\textbf{1}} & \sout{1} & \sout{0} & \sout{1} \\
        & \sout{\textbf{1}} & \sout{1} & \sout{1} & \sout{0} \\
        \hline
    \end{tabular}
    \caption{Macierz blokująca $B_6$} \label{tab:b6d}
\end{table}
Dla macierzy blokującej przedstawionej w Tablicy~\ref{tab:b6d} znajdujemy jedno minimalne pokrycie kolumnowe
$\bm{L'=\{3\}}$ i wynikający z niego implikant prosty $\imp{d}=(1{*}{*}{*})=x_3$.

\begin{table}[H]
    \centering
    \begin{tabular}[t]{ |c|c| }
        \hline
        $I_0$ & ${*}0{*}0$ \\
        $I_1$ & ${*}{*}10$ \\
        \sout{$I_2$} & \sout{${*}0{*}0$} \\
        $I_3$ & ${*}01{*}$ \\
        \sout{$I_4$} & \sout{${*}01{*}$} \\
        $I_5$ & ${*}101$ \\
        \sout{$I_6$} & \sout{${*}{*}10$} \\
        $I_7$ & $1{*}{*}{*}$ \\
        \sout{$I_8$} & \sout{$1{*}{*}{*}$} \\
        \hline
    \end{tabular}
    \caption{Wszystkie implikanty proste} \label{tab:all-implicantsd}
\end{table}
%TODO: Dodać komentarz do tabeki wszytkich implikantów

\begin{table}[H]
    \centering
    \begin{tabular}[t]{ |c||c|c|c|c|c| }
        \hline
        & $I_0 = {*}0{*}0$ & $I_1 = {*}{*}10$ & $I_3 = {*}01{*}$ & $I_5 = {*}101$ & $I_7 = 1{*}{*}{*}$ \\
        \hline
        \hline
        $k_0 = 0000$ & \textbf{1} & 0 & 0 & 0 & 0 \\
        \hline
        \sout{$k_1 = 0010$} &  \sout{\textbf{1}} &  \sout{\textbf{1}} &  \sout{\textbf{1}} & \sout{0} & \sout{0} \\
        \hline
        $k_2 = 0011$ & 0 & 0 & \textbf{1} & 0 & 0 \\
        \hline
        $k_3 = 0101$ & 0 & 0 & 0 & \textbf{1} & 0 \\
        \hline
        $k_4 = 0110$ & 0 & \textbf{1} & 0 & 0 & 0 \\
        \hline
        \sout{$k_5 = 1000$} &  \sout{\textbf{1}} & \sout{0} & \sout{0} & \sout{0} &  \sout{\textbf{1}} \\
        \hline
        $k_6 = 1001$ & 0 & 0 & 0 & 0 & \textbf{1} \\
        \hline
    \end{tabular}
    \caption{} \label{tab:min-blockd}
\end{table}
Minimalne pokrycie kolumnowe implikantów prostych $L' = \{I_0, I_1, I_3, I_5, I_7\}$.
$f_d(x_3, x_2, x_1, x_0) = x_2'x_0' + x_{1}x_0' + x_2'x_1 + x_3$

    \newpage
    \section{Zadanie 5 - Minimalizacja funkcji boolowskiej z wykorzystaniem heurystycznej metody ekspansji}\label{sec:task-4}
    \subsection{Funkcja c}\label{subsec:fun-e}
    %! Suppress = MissingImport
\setcounter{implicant_counter}{0}
%0,2,6,8
Funkcja $f_e$ rozpisana na macierze $F$ i $R$ ma postać:
\begin{center}
    \begin{tabular}[t]{ |c|c c c c|}
        \hline
        $F$ & $x_3$ & $x_2$ & $x_1$ & $x_0$ \\
        \hline
        $k_0$ & 0 & 0 & 0 & 0 \\
        $k_1$ & 0 & 0 & 1 & 0 \\
        $k_2$ & 0 & 1 & 1 & 0 \\
        $k_3$ & 1 & 0 & 0 & 0 \\
        \hline
    \end{tabular}
    \hspace{1cm}
    \begin{tabular}[t]{ |c|c c c c| }
        \hline
        $R$ & $x_3$ & $x_2$ & $x_1$ & $x_0$ \\
        \hline
        & 0 & 0 & 0 & 1 \\
        & 0 & 0 & 1 & 1 \\
        & 0 & 1 & 0 & 0 \\
        & 0 & 1 & 0 & 1 \\
        & 0 & 1 & 1 & 1 \\
        & 1 & 0 & 0 & 1 \\
        \hline
    \end{tabular}
\end{center}

Rozpoczynamy od wyznaczenia wszystkich implikantów prostych korzystając z macierzy blokujących dla kolejnych kostek
$k_0-k_4$.

\begin{table}[H]
    \centering
    \begin{tabular}[t]{ |c|c c c c| }
        \hline
        $k_0$ & 0 & 0 & 0 & 1 \\
        \hline\hline
        $B_0$ & $x_3$ & $x_2$ & $x_1$ & $x_0$ \\
        \hline
        & 0 & 0 & 0 & 1 \\
        & 0 & 0 & 1 & 1 \\
        & 0 & 1 & 0 & 0 \\
        & 0 & 1 & 0 & 1 \\
        & 0 & 1 & 1 & 1 \\
        & 1 & 0 & 0 & 1 \\
        \hline
    \end{tabular}
    \caption{Macierz blokująca $B_0$} \label{tab:b0}
\end{table}

    \newpage
    \subsection{Funkcja f}\label{subsec:fun-f}
    %! Suppress = MissingImport
\setcounter{implicant_counter}{0}

Funkcja $f_e$ rozpisana na macierze $F$ i $R$ ma postać:
\begin{center}
    \begin{tabular}[t]{ |c|c c c c|}
        \hline
        $F$ & $x_3$ & $x_2$ & $x_1$ & $x_0$ \\
        \hline
        $k_0$ & 0 & 0 & 0 & 0 \\
        $k_1$ & 0 & 1 & 0 & 0 \\
        $k_2$ & 0 & 1 & 0 & 1 \\
        $k_3$ & 0 & 1 & 1 & 0 \\
        $k_4$ & 1 & 0 & 0 & 0 \\
        $k_5$ & 1 & 0 & 0 & 1 \\
        \hline
    \end{tabular}
    \hspace{1cm}
    \begin{tabular}[t]{ |c|c c c c| }
        \hline
        $R$ & $x_3$ & $x_2$ & $x_1$ & $x_0$ \\
        \hline
        & 0 & 0 & 0 & 1 \\
        & 0 & 0 & 1 & 0 \\
        & 0 & 0 & 1 & 1 \\
        & 0 & 1 & 1 & 1 \\
        \hline
    \end{tabular}
\end{center}

Rozpoczynamy od wyznaczenia wszystkich implikantów prostych korzystając z macierzy blokujących dla kolejnych kostek
$k_0-k_5$.

\begin{table}[H]
    \centering
    \begin{tabular}[t]{ |c|c c c c| }
        \hline
        $k_0$ & 0 & 0 & 0 & 0 \\
        \hline\hline
        $B_0$ & $x_3$ & $x_2$ & $x_1$ & $x_0$ \\
        \hline
        & 0 & 0 & 0 & \textbf{1} \\
        & 0 & 0 & \textbf{1} & 0 \\
        & \sout{0} & \sout{0} & \sout{1} & \sout{1} \\
        & \sout{0} & \sout{1} & \sout{1} & \sout{1} \\
        \hline
    \end{tabular}
    \caption{Macierz blokująca $B_0$}\label{tab:b0f}
\end{table}

Dla macierzy blokującej przedstawionej w Tablicy~\ref{tab:b0f} znajdujemy jedno minimalne pokrycie kolumnowe
$\bm{L'=\{1,0\}}$ i wynikający z niego implikant prosty $\imp=({*}{*}00)=x_0'x_1'$.

\begin{table}[H]
    \centering
    \begin{tabular}[t]{ |c|c c c c|}
        \hline
        $I_0$ & * & * & 0 & 0 \\
        \hline\hline
        $F$ & $x_3$ & $x_2$ & $x_1$ & $x_0$ \\
        \hline
        \sout{$k_0$} & \sout{0} & \sout{0} & \sout{0} & \sout{0} \\
        \sout{$k_1$} & \sout{0} & \sout{1} & \sout{0} & \sout{0} \\
        $k_2$ & 0 & 1 & 0 & 1 \\
        $k_3$ & 0 & 1 & 1 & 0 \\
        \sout{$k_4$} & \sout{1} & \sout{0} & \sout{0} & \sout{0} \\
        $k_5$ & 1 & 0 & 0 & 1 \\
        \hline
    \end{tabular}
    \caption{Kostki do wykreślenia}\label{tab:die-0f}
\end{table}
%TODO: Dodać komentarz do tabelki wykreślającej

\begin{table}[H]
    \centering
    \begin{tabular}[t]{ |c|c c c c| }
        \hline
        $k_2$ & 0 & 1 & 0 & 1 \\
        \hline\hline
        $B_1$ & $x_3$ & $x_2$ & $x_1$ & $x_0$ \\
        \hline
        & 0 & \textbf{1} & 0 & 0 \\
        & \sout{0} & \sout{1} & \sout{1} & \sout{1} \\
        & \sout{0} & \sout{1} & \sout{1} & \sout{0} \\
        & 0 & 0 & \textbf{1} & 0 \\
        \hline
    \end{tabular}
    \caption{Macierz blokująca $B_1$}\label{tab:b1f}
\end{table}

Dla macierzy blokującej przedstawionej w Tablicy~\ref{tab:b1f} znajdujemy jedno minimalne pokrycie kolumnowe
$\bm{L'=\{2,1\}}$ i wynikający z niego implikant prosty $\imp=({*}10{*})=x_1'x_2$.

\begin{table}[H]
    \centering
    \begin{tabular}[t]{ |c|c c c c|}
        \hline
        $I_1$ & * & 1 & 0 & * \\
        \hline\hline
        $F$ & $x_3$ & $x_2$ & $x_1$ & $x_0$ \\
        \hline
        \sout{$k_2$} & \sout{0} & \sout{1} & \sout{0} & \sout{1} \\
        $k_3$ & 0 & 1 & 1 & 0 \\
        $k_5$ & 1 & 0 & 0 & 1 \\
        \hline
    \end{tabular}
    \caption{Kostki do wykreślenia}\label{tab:die-1f}
\end{table}
%TODO: Dodać komentarz do tabelki wykreślającej

\begin{table}[H]
    \centering
    \begin{tabular}[t]{ |c|c c c c| }
        \hline
        $k_3$ & 0 & 1 & 1 & 0 \\
        \hline\hline
        $B_2$ & $x_3$ & $x_2$ & $x_1$ & $x_0$ \\
        \hline
        & \sout{0} & \sout{1} & \sout{1} & \sout{1} \\
        & 0 & \textbf{1} & 0 & 0 \\
        & \sout{0} & \sout{1} & \sout{0} & \sout{1} \\
        & 0 & 0 & 0 & \textbf{1} \\
        \hline
    \end{tabular}
    \caption{Macierz blokująca $B_2$}\label{tab:b2f}
\end{table}

Dla macierzy blokującej przedstawionej w Tablicy~\ref{tab:b2f} znajdujemy jedno minimalne pokrycie kolumnowe
$\bm{L'=\{2,0\}}$ i wynikający z niego implikant prosty $\imp=({*}1{*}0)=x_0'x_2$.

\begin{table}[H]
    \centering
    \begin{tabular}[t]{ |c|c c c c|}
        \hline
        $I_2$ & * & 1 & * & 0 \\
        \hline\hline
        $F$ & $x_3$ & $x_2$ & $x_1$ & $x_0$ \\
        \hline
        \sout{$k_3$} & \sout{0} & \sout{1} & \sout{1} & \sout{0} \\
        $k_5$ & 1 & 0 & 0 & 1 \\
        \hline
    \end{tabular}
    \caption{Kostki do wykreślenia}\label{tab:die-2f}
\end{table}
%TODO: Dodać komentarz do tabelki wykreślającej

\begin{table}[H]
    \centering
    \begin{tabular}[t]{ |c|c c c c| }
        \hline
        $k_5$ & 1 & 0 & 0 & 1 \\
        \hline\hline
        $B_3$ & $x_3$ & $x_2$ & $x_1$ & $x_0$ \\
        \hline
        & \textbf{1} & 0 & 0 & 0 \\
        & \sout{1} & \sout{0} & \sout{1} & \sout{1} \\
        & \sout{1} & \sout{0} & \sout{1} & \sout{0} \\
        & \sout{1} & \sout{1} & \sout{1} & \sout{0} \\
        \hline
    \end{tabular}
    \caption{Macierz blokująca $B_3$}\label{tab:b3f}
\end{table}

Dla macierzy blokującej przedstawionej w Tablicy~\ref{tab:b3f} znajdujemy jedno minimalne pokrycie kolumnowe
$\bm{L'=\{3\}}$ i wynikający z niego implikant prosty $\imp=(1{*}{*}{*})=x_3$.

\begin{table}[H]
    \centering
    \begin{tabular}[t]{ |c|c c c c|}
        \hline
        $I_3$ & * & 1 & * & 0 \\
        \hline\hline
        $F$ & $x_3$ & $x_2$ & $x_1$ & $x_0$ \\
        \hline
        \sout{$k_5$} & \sout{1} & \sout{0} & \sout{0} & \sout{1} \\
        \hline
    \end{tabular}
    \caption{Kostki do wykreślenia}\label{tab:die-3f}
\end{table}
%TODO: Dodać komentarz do tabelki wykreślającej

$f_d(x_3, x_2, x_1, x_0) = I_0 + I_1 + I_2 + I_3 = x_0'x_1' + x_1'x_2 + x_0'x_2 + x_3$

\end{document}
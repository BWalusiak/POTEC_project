%! Author = Bartosz Walusiak
%! Date = 11/16/2019

% Preamble
\documentclass[a4paper, 12pt]{article}

\usepackage{float}
\usepackage[T1]{fontenc}
\usepackage{lmodern}
\usepackage[polish]{babel}
\usepackage[utf8]{inputenc}
\usepackage{polski}
\usepackage{hyperref}
\usepackage{karnaugh-map}

\begin{document}
    \title{Projekt nr 1 POTEC}
    \author{Monika Lewandowska, Bartosz Walusiak, Krzysztof Zdulski}
    \date{\today}
    \maketitle

    \newpage
    \section{Zadanie 1 - Minimalizacja funkcji za pomocą tablic Karnaugha dla mintermów}\label{sec:task-1}
    \subsection{Funckja g}\label{subsec:fun-g}
    Funkcja \(f_g\) zapisana w postaci mintermów wygląda następująco.
    \[f_g(x_3, x_2, x_1, x_0) = \sum (2, 3, 4, 5, 6, 8, 9, (10, 11, 12, 13, 14, 15))\]
    \begin{figure}[H]
        \centering
        \begin{karnaugh-map}[4][4][1][$x_1x_0$][$x_3x_2$]
            \minterms{2, 3, 4, 5, 6, 8, 9}
            \maxterms{0, 1, 7}
            \indeterminants{10, 11, 12, 13, 14, 15}
            \implicant{2}{10}
            \implicant{12}{10}
            \implicant{4}{13}
            \implicantedge{3}{2}{11}{10}
        \end{karnaugh-map}
        \caption{Tablica dla funkcji \textrm{g}}
        \label{fig:fg}
    \end{figure}
    Minimalizację funkcji \(f_g(x_3, x_2, x_1, x_0)\) za pomocą tablicy Karnaugha przedstawiono na Rys.~\ref{fig:fg}.
    Znaleziono 4 implikanty proste, z których wszystkie były niezbędne (\textrm{EIP}).
    \begin{equation}\label{eq:fg}
        f_g(x_3, x_2, x_1, x_0) = x_3 + x_0'x_1 + x_1x_2' + x_1'x_2
    \end{equation}
    Równanie~\ref{eq:fg} jest zminimalizowanym równaniem funkcji  \(f_g\).

    \newpage
    \section{Zadanie 2 - Minimalizacja funkcji za pomocą tablic Karnaugha dla maxtermów}\label{sec:task-2}
    \subsection{Funckja a}\label{subsec:fun-a}
    \[f_a(x_3, x_2, x_1, x_0) = \prod (1, 4, (10, 11, 12, 13, 14, 15))\]
    \begin{figure}[h]
        \centering
        \begin{karnaugh-map}[4][4][1][$x_1x_0$][$x_3x_2$]
            \minterms{0, 2, 3, 5, 6, 7, 8, 9}
            \maxterms{1, 4}
            \indeterminants{10, 11, 12, 13, 14, 15}
            \implicant{4}{12}
            \implicant{1}{1}
        \end{karnaugh-map}
        \caption{Tabela dla funkcji \textrm{a}}
        \label{fig:fa}
    \end{figure}
    \begin{equation}\label{eq:fa}
        (x_0 + x_1 + x_2')(x_0' + x_1 + x_2 + x_3)
    \end{equation}

    \newpage
    \subsection{Funckja b}\label{subsec:fun-b}
    \[f_b(x_3, x_2, x_1, x_0) = \prod (5, 6, (10, 11, 12, 13, 14, 15))\]
    \begin{figure}[h]
        \centering
        \begin{karnaugh-map}[4][4][1][$x_1x_0$][$x_3x_2$]
            \minterms{0, 1, 2, 3, 4, 7, 8, 9}
            \maxterms{5, 6}
            \indeterminants{10, 11, 12, 13, 14, 15}
            \implicant{5}{13}
            \implicant{6}{14}
        \end{karnaugh-map}
        \caption{Tabela dla funkcji \textrm{b}}
        \label{fig:fb}
    \end{figure}
    \begin{equation}
        (x_0' + x_1 + x_2')(x_0 + x_1' + x_2')
    \end{equation}
\end{document}
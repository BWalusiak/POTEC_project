%! Suppress = MissingImport
\setcounter{implicant_counter}{0}

Funkcja $f_e$ rozpisana na macierze $F$ i $R$ ma postać:
\begin{center}
    \begin{tabular}[t]{ |c|c c c c|}
        \hline
        $F$ & $x_3$ & $x_2$ & $x_1$ & $x_0$ \\
        \hline
        $k_0$ & 0 & 0 & 0 & 0 \\
        $k_1$ & 0 & 1 & 0 & 0 \\
        $k_2$ & 0 & 1 & 0 & 1 \\
        $k_3$ & 0 & 1 & 1 & 0 \\
        $k_4$ & 1 & 0 & 0 & 0 \\
        $k_5$ & 1 & 0 & 0 & 1 \\
        \hline
    \end{tabular}
    \hspace{1cm}
    \begin{tabular}[t]{ |c|c c c c| }
        \hline
        $R$ & $x_3$ & $x_2$ & $x_1$ & $x_0$ \\
        \hline
        & 0 & 0 & 0 & 1 \\
        & 0 & 0 & 1 & 0 \\
        & 0 & 0 & 1 & 1 \\
        & 0 & 1 & 1 & 1 \\
        \hline
    \end{tabular}
\end{center}

Rozpoczynamy od wyznaczenia wszystkich implikantów prostych korzystając z macierzy blokujących dla kolejnych kostek
$k_0-k_5$.

\begin{table}[H]
    \centering
    \begin{tabular}[t]{ |c|c c c c| }
        \hline
        $k_0$ & 0 & 0 & 0 & 0 \\
        \hline\hline
        $B_0$ & $x_3$ & $x_2$ & $x_1$ & $x_0$ \\
        \hline
        & 0 & 0 & 0 & \textbf{1} \\
        & 0 & 0 & \textbf{1} & 0 \\
        & \sout{0} & \sout{0} & \sout{1} & \sout{1} \\
        & \sout{0} & \sout{1} & \sout{1} & \sout{1} \\
        \hline
    \end{tabular}
    \caption{Macierz blokująca $B_0$}\label{tab:b0f}
\end{table}

Dla macierzy blokującej przedstawionej w Tablicy~\ref{tab:b0f} znajdujemy jedno minimalne pokrycie kolumnowe
$\bm{L'=\{1,0\}}$ i wynikający z niego implikant prosty $\imp{f}=({*}{*}00)=x_0'x_1'$.

\begin{table}[H]
    \centering
    \begin{tabular}[t]{ |c|c c c c|}
        \hline
        $I_0$ & * & * & 0 & 0 \\
        \hline\hline
        $F$ & $x_3$ & $x_2$ & $x_1$ & $x_0$ \\
        \hline
        \sout{$k_0$} & \sout{0} & \sout{0} & \sout{0} & \sout{0} \\
        \sout{$k_1$} & \sout{0} & \sout{1} & \sout{0} & \sout{0} \\
        $k_2$ & 0 & 1 & 0 & 1 \\
        $k_3$ & 0 & 1 & 1 & 0 \\
        \sout{$k_4$} & \sout{1} & \sout{0} & \sout{0} & \sout{0} \\
        $k_5$ & 1 & 0 & 0 & 1 \\
        \hline
    \end{tabular}
    \caption{Kostki do wykreślenia}\label{tab:die-0f}
\end{table}
%TODO: Dodać komentarz do tabelki wykreślającej

\begin{table}[H]
    \centering
    \begin{tabular}[t]{ |c|c c c c| }
        \hline
        $k_2$ & 0 & 1 & 0 & 1 \\
        \hline\hline
        $B_1$ & $x_3$ & $x_2$ & $x_1$ & $x_0$ \\
        \hline
        & 0 & \textbf{1} & 0 & 0 \\
        & \sout{0} & \sout{1} & \sout{1} & \sout{1} \\
        & \sout{0} & \sout{1} & \sout{1} & \sout{0} \\
        & 0 & 0 & \textbf{1} & 0 \\
        \hline
    \end{tabular}
    \caption{Macierz blokująca $B_1$}\label{tab:b1f}
\end{table}

Dla macierzy blokującej przedstawionej w Tablicy~\ref{tab:b1f} znajdujemy jedno minimalne pokrycie kolumnowe
$\bm{L'=\{2,1\}}$ i wynikający z niego implikant prosty $\imp{f}=({*}10{*})=x_1'x_2$.

\begin{table}[H]
    \centering
    \begin{tabular}[t]{ |c|c c c c|}
        \hline
        $I_1$ & * & 1 & 0 & * \\
        \hline\hline
        $F$ & $x_3$ & $x_2$ & $x_1$ & $x_0$ \\
        \hline
        \sout{$k_2$} & \sout{0} & \sout{1} & \sout{0} & \sout{1} \\
        $k_3$ & 0 & 1 & 1 & 0 \\
        $k_5$ & 1 & 0 & 0 & 1 \\
        \hline
    \end{tabular}
    \caption{Kostki do wykreślenia}\label{tab:die-1f}
\end{table}
%TODO: Dodać komentarz do tabelki wykreślającej

\begin{table}[H]
    \centering
    \begin{tabular}[t]{ |c|c c c c| }
        \hline
        $k_3$ & 0 & 1 & 1 & 0 \\
        \hline\hline
        $B_2$ & $x_3$ & $x_2$ & $x_1$ & $x_0$ \\
        \hline
        & \sout{0} & \sout{1} & \sout{1} & \sout{1} \\
        & 0 & \textbf{1} & 0 & 0 \\
        & \sout{0} & \sout{1} & \sout{0} & \sout{1} \\
        & 0 & 0 & 0 & \textbf{1} \\
        \hline
    \end{tabular}
    \caption{Macierz blokująca $B_2$}\label{tab:b2f}
\end{table}

Dla macierzy blokującej przedstawionej w Tablicy~\ref{tab:b2f} znajdujemy jedno minimalne pokrycie kolumnowe
$\bm{L'=\{2,0\}}$ i wynikający z niego implikant prosty $\imp{f}=({*}1{*}0)=x_0'x_2$.

\begin{table}[H]
    \centering
    \begin{tabular}[t]{ |c|c c c c|}
        \hline
        $I_2$ & * & 1 & * & 0 \\
        \hline\hline
        $F$ & $x_3$ & $x_2$ & $x_1$ & $x_0$ \\
        \hline
        \sout{$k_3$} & \sout{0} & \sout{1} & \sout{1} & \sout{0} \\
        $k_5$ & 1 & 0 & 0 & 1 \\
        \hline
    \end{tabular}
    \caption{Kostki do wykreślenia}\label{tab:die-2f}
\end{table}
%TODO: Dodać komentarz do tabelki wykreślającej

\begin{table}[H]
    \centering
    \begin{tabular}[t]{ |c|c c c c| }
        \hline
        $k_5$ & 1 & 0 & 0 & 1 \\
        \hline\hline
        $B_3$ & $x_3$ & $x_2$ & $x_1$ & $x_0$ \\
        \hline
        & \textbf{1} & 0 & 0 & 0 \\
        & \sout{1} & \sout{0} & \sout{1} & \sout{1} \\
        & \sout{1} & \sout{0} & \sout{1} & \sout{0} \\
        & \sout{1} & \sout{1} & \sout{1} & \sout{0} \\
        \hline
    \end{tabular}
    \caption{Macierz blokująca $B_3$}\label{tab:b3f}
\end{table}

Dla macierzy blokującej przedstawionej w Tablicy~\ref{tab:b3f} znajdujemy jedno minimalne pokrycie kolumnowe
$\bm{L'=\{3\}}$ i wynikający z niego implikant prosty $\imp{f}=(1{*}{*}{*})=x_3$.

\begin{table}[H]
    \centering
    \begin{tabular}[t]{ |c|c c c c|}
        \hline
        $I_3$ & * & 1 & * & 0 \\
        \hline\hline
        $F$ & $x_3$ & $x_2$ & $x_1$ & $x_0$ \\
        \hline
        \sout{$k_5$} & \sout{1} & \sout{0} & \sout{0} & \sout{1} \\
        \hline
    \end{tabular}
    \caption{Kostki do wykreślenia}\label{tab:die-3f}
\end{table}
%TODO: Dodać komentarz do tabelki wykreślającej

$f_d(x_3, x_2, x_1, x_0) = I_0 + I_1 + I_2 + I_3 = x_0'x_1' + x_1'x_2 + x_0'x_2 + x_3$
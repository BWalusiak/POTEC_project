%! Suppress = MissingImport
\setcounter{implicant_counter}{0}

Funkcja $f_e$ rozpisana na macierze $F$ i $R$ ma postać:
\begin{center}
    \begin{tabular}[t]{ |c|c c c c| }
        \hline
        $F$ & $x_3$ & $x_2$ & $x_1$ & $x_0$ \\
        \hline
        $k_0$ & 0 & 0 & 0 & 1 \\
        $k_1$ & 0 & 0 & 1 & 0 \\
        $k_2$ & 0 & 0 & 1 & 1 \\
        $k_3$ & 0 & 1 & 1 & 1 \\
        \hline
    \end{tabular}
    \hspace{1cm}
    \begin{tabular}[t]{ |c|c c c c| }
        \hline
        $R$ & $x_3$ & $x_2$ & $x_1$ & $x_0$ \\
        \hline
        & 0 & 0 & 0 & 0 \\
        & 0 & 1 & 0 & 0 \\
        & 0 & 1 & 0 & 1 \\
        & 0 & 1 & 1 & 0 \\
        & 1 & 0 & 0 & 0 \\
        & 1 & 0 & 0 & 1 \\
        \hline
    \end{tabular}
\end{center}

Rozpoczynamy od wyznaczenia wszystkich implikantów prostych korzystając z macierzy blokujących dla kolejnych kostek
$k_0-k_3$.

\begin{table}[H]
    \centering
    \begin{tabular}[t]{ |c|c c c c| }
        \hline
        $k_0$ & 0 & 0 & 0 & 1 \\
        \hline\hline
        $B_0$ & $x_3$ & $x_2$ & $x_1$ & $x_0$ \\
        \hline
        & 0 & 0 & 0 & \textbf{1} \\
        & \sout{0} & \sout{\textbf{1}} & \sout{0} & \sout{\textbf{1}} \\
        & 0 & \textbf{1} & 0 & 0 \\
        & \sout{0} & \sout{\textbf{1}} & \sout{1} & \sout{\textbf{1}} \\
        & \sout{\textbf{1}} & \sout{0} & \sout{0} & \sout{\textbf{1}} \\
        & \textbf{1} & 0 & 0 & 0 \\
        \hline
    \end{tabular}
    \caption{Macierz blokująca $B_0$}\label{tab:b0f}
\end{table}

Dla macierzy blokującej przedstawionej w Tablicy~\ref{tab:b0f} znajdujemy jedno minimalne pokrycie kolumnowe
$\bm{L'=\{3,2,0\}}$ i wynikający z niego implikant prosty $\imp{f}=(00{*}1)=x_3'x_2'x_0$.

\begin{table}[H]
    \centering
    \begin{tabular}[t]{ |c|c c c c| }
        \hline
        $I_0$ & 0 & 0 & $*$ & 1 \\
        \hline\hline
        $F$ & $x_3$ & $x_2$ & $x_1$ & $x_0$ \\
        \hline
        \sout{$k_0$} & \sout{0} & \sout{0} & \sout{0} & \sout{1} \\
        $k_1$ & 0 & 0 & 1 & 0 \\
        \sout{$k_2$} & \sout{0} & \sout{0} & \sout{1} & \sout{1} \\
        $k_3$ & 0 & 1 & 1 & 1 \\
        \hline
    \end{tabular}
    \caption{Kostki do wykreślenia}\label{tab:die-0f}
\end{table}
W Tablicy~\ref{tab:die-0f} przedstawiono kostki z macierzy $F$, które odpowiadają znalezionemu implikantowi $I_0$
i które należy wykreślić.

\begin{table}[H]
    \centering
    \begin{tabular}[t]{ |c|c c c c| }
        \hline
        $k_1$ & 0 & 0 & 1 & 0 \\
        \hline\hline
        $B_1$ & $x_3$ & $x_2$ & $x_1$ & $x_0$ \\
        \hline
        & 0 & 0 & \textbf{1} & 0 \\
        & \sout{0} & \sout{\textbf{1}} & \sout{\textbf{1}} & \sout{0} \\
        & \sout{0} & \sout{\textbf{1}} & \sout{\textbf{1}} & \sout{1} \\
        & 0 & \textbf{1} & 0 & 0 \\
        & \sout{1} & \sout{0} & \sout{\textbf{1}} & \sout{0} \\
        & \sout{1} & \sout{0} & \sout{\textbf{1}} & \sout{1} \\
        \hline
    \end{tabular}
    \caption{Macierz blokująca $B_1$}\label{tab:b1f}
\end{table}

Dla macierzy blokującej przedstawionej w Tablicy~\ref{tab:b1f} znajdujemy jedno minimalne pokrycie kolumnowe
$\bm{L'=\{2,1\}}$ i wynikający z niego implikant prosty $\imp{f}=({*}01{*})=x_2'x_1$.

\begin{table}[H]
    \centering
    \begin{tabular}[t]{ |c|c c c c|}
        \hline
        $I_1$ & $*$ & 0 & 1 & $*$ \\
        \hline\hline
        $F$ & $x_3$ & $x_2$ & $x_1$ & $x_0$ \\
        \hline
        \sout{$k_1$} & \sout{0} & \sout{0} & \sout{1} & \sout{0} \\
        $k_3$ & 0 & 1 & 1 & 1 \\
        \hline
    \end{tabular}
    \caption{Kostki do wykreślenia}\label{tab:die-1f}
\end{table}
W Tablicy~\ref{tab:die-1f} przedstawiono kostki z macierzy $F$, które odpowiadają znalezionemu implikantowi $I_1$
i które należy wykreślić.

\begin{table}[H]
    \centering
    \begin{tabular}[t]{ |c|c c c c| }
        \hline
        $k_3$ & 0 & 1 & 1 & 1 \\
        \hline\hline
        $B_2$ & $x_3$ & $x_2$ & $x_1$ & $x_0$ \\
        \hline
        & \sout{0} & \sout{1} & \sout{\textbf{1}} & \sout{\textbf{1}} \\
        & \sout{0} & \sout{0} & \sout{\textbf{1}} & \sout{\textbf{1}} \\
        & 0 & 0 & \textbf{1} & 0 \\
        & 0 & 0 & 0 & \textbf{1} \\
        & \sout{1} & \sout{1} & \sout{\textbf{1}} & \sout{\textbf{1}} \\
        & \sout{1} & \sout{1} & \sout{\textbf{1}} & \sout{0} \\
        \hline
    \end{tabular}
    \caption{Macierz blokująca $B_2$}\label{tab:b2f}
\end{table}

Dla macierzy blokującej przedstawionej w Tablicy~\ref{tab:b2f} znajdujemy jedno minimalne pokrycie kolumnowe
$\bm{L'=\{1,0\}}$ i wynikający z niego implikant prosty $\imp{f}=({*}{*}11)=x_{1}x_0$.

\begin{table}[H]
    \centering
    \begin{tabular}[t]{ |c|c c c c|}
        \hline
        $I_2$ & * & 1 & * & 0 \\
        \hline\hline
        $F$ & $x_3$ & $x_2$ & $x_1$ & $x_0$ \\
        \hline
        \sout{$k_3$} & \sout{0} & \sout{1} & \sout{1} & \sout{1} \\
        \hline
    \end{tabular}
    \caption{Kostki do wykreślenia}\label{tab:die-2f}
\end{table}
W Tablicy~\ref{tab:die-2f} przedstawiono kostki z macierzy $F$, które odpowiadają znalezionemu implikantowi $I_2$
i które należy wykreślić.

Ostateczną postać funkcji $f_f$, po minimalizacji przedstawiono w równaniu~\ref{eq:ff}.
\begin{equation}
    \label{eq:ff}
    f_f(x_3, x_2, x_1, x_0) = I_0 + I_1 + I_2 = x_3'x_2'x_0 + x_2'x_1 + x_{1}x_0
\end{equation}

Projekt ma na celu analizę i realizację 7-segmentowego układu obwodu dekodera. Wyświetlacze siedmiosegmentowe są tymi najczęściej stosowanymi do wyświetlenia znaków i liczb (w celu uzyskania odczytu dziesiętnego).
Pierwszym etapem projektu było dokonanie minimalizacji funkcji boolowskich różnymi metodami takimi jak: metoda tabic Karnaugha zarówno dla minterów jak i dla makstermów oraz metoda ekspansji systematyczna i heurystyczna.
Wykorzystaliśmy tablicę opisującą działanie dekodera, gdzie aktywnym synałem jest 0.
Drugim etapem projektu była realizacja tego układu dekodera w postaci sieci bramek dla minimalnych równań w programie Logisim.

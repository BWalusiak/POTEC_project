\subsection{Cel projektu}\label{subsec:intro-goal}
Projekt ma na celu analizę i realizację siedmiosegmentowego układu obwodu dekodera, którego aktywnym sygnałem jest $0$.
Wyświetlacze siedmiosegmentowe są tymi najczęściej stosowanymi do
wyświetlenia liczb (w celu uzyskania odczytu dziesiętnego).

\subsection{Realizacja projektu}\label{subsec:intro-how}
Realizacja układu dekodera została podzielona na 3 etapy:
\begin{enumerate}
    \item Zapisano tablicę prawdy dla każdego z segmentów - przedstawioną w Tablicy~\ref{tab:truth-table}.
    \item Wyznaczono minimalne postaci każdej z funkcji boolowskich korzystając
          z metod minimalizacji tablic Karnaugha oraz Espresso.
    \item Zrealizowano dekoder przy użyciu oprogramowania Logisim, mając na uwadze fakt,
          iż aktywnym sygnałem w tym programie jest $1$.
\end{enumerate}

\begin{table}[H]
    \centering
    \begin{tabular}{|c||c|c|c|c||c|c|c|c|c|c|c|}
        \hline
        Cyfra & $x_3$ & $x_2$ & $x_1$ & $x_0$ & $f_a$ & $f_b$ & $f_c$ & $f_d$ & $f_e$ & $f_f$ & $f_g$ \\ \hline
        \hline
        0     & 0     & 0     & 0     & 0     & 0     & 0     & 0     & 0     & 0     & 0     & 1     \\ \hline
        1     & 0     & 0     & 0     & 1     & 1     & 0     & 0     & 1     & 1     & 1     & 1     \\ \hline
        2     & 0     & 0     & 1     & 0     & 0     & 0     & 1     & 0     & 0     & 1     & 0     \\ \hline
        3     & 0     & 0     & 1     & 1     & 0     & 0     & 0     & 0     & 1     & 1     & 0     \\ \hline
        4     & 0     & 1     & 0     & 0     & 1     & 0     & 0     & 1     & 1     & 0     & 0     \\ \hline
        5     & 0     & 1     & 0     & 1     & 0     & 1     & 0     & 0     & 1     & 0     & 0     \\ \hline
        6     & 0     & 1     & 1     & 0     & 0     & 1     & 0     & 0     & 0     & 0     & 0     \\ \hline
        7     & 0     & 1     & 1     & 1     & 0     & 0     & 0     & 1     & 1     & 1     & 1     \\ \hline
        8     & 1     & 0     & 0     & 0     & 0     & 0     & 0     & 0     & 0     & 0     & 0     \\ \hline
        9     & 1     & 0     & 0     & 1     & 0     & 0     & 0     & 0     & 1     & 0     & 0     \\ \hline
    \end{tabular}
    \caption{Tablica prawdy dekodera}
    \label{tab:truth-table}
\end{table}
Funkcja $f_a$ zapisana w postaci makstermów wygląda następująco:
\[f_a(x_3, x_2, x_1, x_0) = \prod (0, 2, 3, 5, 6, 7, 8, 9, (10, 11, 12, 13, 14, 15))\]
\begin{figure}[h]
    \centering
    \begin{karnaugh-map}[4][4][1][$x_{1}x_0$][$x_{3}x_2$]
        \minterms{1,4}
        \maxterms{0, 2, 3, 5, 6, 7, 8, 9}
        \indeterminants{10, 11, 12, 13, 14, 15}
        \implicantcorner
        \implicant{12}{10}
        \implicant{3}{10}
        \implicant{5}{15}
        \end{karnaugh-map}
    \caption{Tablica dla funkcji \textrm{a}}
    \label{fig:fa}
\end{figure}
Minimalizację funkcji $f_a(x_3, x_2, x_1, x_0)$ za pomocą tablicy Karnaugha przedstawiono na Rys.~\ref{fig:fa}.
Znaleziono 4 implikanty proste, z których wszystkie są niezbędne (\textrm{EIP}).
\begin{equation}
    \label{eq:fa}
    f_a(x_3, x_2, x_1, x_0) =  x_3' x_1'  (x_2 + x_0) (x_2'+x_0')
\end{equation}
Równanie~\ref{eq:fa} jest zminimalizowanym równaniem funkcji $f_a$.

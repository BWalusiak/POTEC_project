Funkcja $f_g$ zapisana w postaci mintermów wygląda następująco.
\[f_g(x_3, x_2, x_1, x_0) = \sum (2, 3, 4, 5, 6, 8, 9, (10, 11, 12, 13, 14, 15))\]
\begin{figure}[H]
    \centering
    \begin{karnaugh-map}[4][4][1][$x_1x_0$][$x_3x_2$]
        \minterms{2, 3, 4, 5, 6, 8, 9}
        \maxterms{0, 1, 7}
        \indeterminants{10, 11, 12, 13, 14, 15}
        \implicant{2}{10}
        \implicant{12}{10}
        \implicant{4}{13}
        \implicantedge{3}{2}{11}{10}
    \end{karnaugh-map}
    \caption{Tablica dla funkcji \textrm{g}}
    \label{fig:fg}
\end{figure}
Minimalizację funkcji $f_g(x_3, x_2, x_1, x_0)$ za pomocą tablicy Karnaugha przedstawiono na Rys.~\ref{fig:fg}.
Znaleziono 4 implikanty proste, z których wszystkie były niezbędne (\textrm{EIP}).
\begin{equation}
    \label{eq:fg}
    f_g(x_3, x_2, x_1, x_0) = x_3 + x_0'x_1 + x_1x_2' + x_1'x_2
\end{equation}
Równanie~\ref{eq:fg} jest zminimalizowanym równaniem funkcji  $f_g$.
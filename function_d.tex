Funkcja $f_d$ rozpisana na macierze $F$ i $R$ ma postać:
\begin{center}
    \begin{tabular}[t]{ |c|c c c c| }
        \hline
        $F$ & $x_3$ & $x_2$ & $x_1$ & $x_0$ \\
        \hline
        $k_0$ & 0 & 0 & 0 & 0 \\
        $k_1$ & 0 & 0 & 1 & 0 \\
        $k_2$ & 0 & 0 & 1 & 1 \\
        $k_3$ & 0 & 1 & 0 & 1 \\
        $k_4$ & 0 & 1 & 1 & 0 \\
        $k_5$ & 1 & 0 & 0 & 0 \\
        $k_6$ & 1 & 0 & 0 & 1 \\
        \hline
    \end{tabular}
    \hspace{1cm}
    \begin{tabular}[t]{ |c|c c c c| }
        \hline
        $R$ & $x_3$ & $x_2$ & $x_1$ & $x_0$ \\
        \hline
        & 0 & 0 & 0 & 1 \\
        & 0 & 1 & 0 & 0 \\
        & 0 & 1 & 1 & 1 \\
        \hline
    \end{tabular}
\end{center}

Rozpoczynamy od wyznaczenia wszystkich implikantów prostych korzystając z macierzy blokujących dla kolejnych kostek
$k_0-k_6$.
\begin{table}[H]
    \centering
    \begin{tabular}[t]{ |c|c c c c| }
        \hline
        $k_0$ & 0 & 0 & 0 & 0 \\
        \hline\hline
        $B_0$ & $x_3$ & $x_2$ & $x_1$ & $x_0$ \\
        \hline
        & 0 & 0 & 0 & \textbf{1} \\
        & 0 & \textbf{1} & 0 & 0 \\
        & \sout{0} & \sout{1} & \sout{1} & \sout{1} \\
        \hline
    \end{tabular}
    \caption{Macierz blokująca $B_0$} \label{tab:b0}
\end{table}
Dla macierzy blokującej przedstawionej w Tablicy~\ref{tab:b0} znajdujemy jedno minimalne pokrycie kolumnowe
$\bm{L'=\{2,0\}}$ i wynikający z niego implikant prosty $I_0=(*0*0)=x_2'x_0'$.

\begin{table}[H]
    \centering
    \begin{tabular}[t]{ |c|c c c c| }
        \hline
        $k_1$ & 0 & 0 & 1 & 0 \\
        \hline\hline
        $B_1$ & $x_3$ & $x_2$ & $x_1$ & $x_0$ \\
        \hline
        & 0 & 0 & \textbf{1} & \textbf{1} \\
        & 0 & 1 & \textbf{1} & 0 \\
        & 0 & 1 & 0 & \textbf{1} \\
        \hline
    \end{tabular}
    \caption{Macierz blokująca $B_1$} \label{tab:b1}
\end{table}
Dla macierzy blokującej przedstawionej w Tablicy~\ref{tab:b1} znajdujemy trzy minimalne pokrycia kolumnowe
$\bm{L'=\{1,0\}}$, $L'=\{2,0\}$, $L'=\{2,1\}$ i
wynikający z nich implikanty proste $I_1=(**10)=x_{1}x_0'$, \sout{$I_2=(*0*0)=x_2'x_0'$}, $I_3=(*01*)=x_2'x_1$.
Zauważamy, że implikant $I_2$ jest identyczny do implikantu $I_0$, więc go wykreślamy.
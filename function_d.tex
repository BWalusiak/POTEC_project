%! Suppress = MissingImport
\setcounter{implicant_counter}{0}

Funkcja $f_d$ rozpisana na macierze $F$ i $R$ ma postać:
\begin{center}
    \begin{tabular}[t]{ |c|c c c c| }
        \hline
        $F$ & $x_3$ & $x_2$ & $x_1$ & $x_0$ \\
        \hline
        $k_0$ & 0 & 0 & 0 & 1 \\
        $k_1$ & 0 & 1 & 0 & 0 \\
        $k_2$ & 0 & 1 & 1 & 1 \\
        \hline
    \end{tabular}
    \hspace{1cm}
    \begin{tabular}[t]{ |c|c c c c| }
        \hline
        $R$ & $x_3$ & $x_2$ & $x_1$ & $x_0$ \\
        \hline
        & 0 & 0 & 0 & 0 \\
        & 0 & 0 & 1 & 0 \\
        & 0 & 0 & 1 & 1 \\
        & 0 & 1 & 0 & 1 \\
        & 0 & 1 & 1 & 0 \\
        & 1 & 0 & 0 & 0 \\
        & 1 & 0 & 0 & 1 \\
        \hline
    \end{tabular}
\end{center}

Rozpoczynamy od wyznaczenia wszystkich implikantów prostych korzystając z macierzy blokujących dla kolejnych kostek
$k_0-k_2$.
\begin{table}[H]
    \centering
    \begin{tabular}[t]{ |c|c c c c| }
        \hline
        $k_0$ & 0 & 0 & 0 & 1 \\
        \hline\hline
        $B_0$ & $x_3$ & $x_2$ & $x_1$ & $x_0$ \\
        \hline
        & 0 & 0 & 0 & \textbf{1} \\
        & \sout{0} & \sout{0} & \sout{1} & \sout{1} \\
        & 0 & 0 & \textbf{1} & 0 \\
        & 0 & \textbf{1} & 0 & 0 \\
        & \sout{0} & \sout{1} & \sout{1} & \sout{1} \\
        & \sout{1} & \sout{0} & \sout{0} & \sout{1} \\
        & \textbf{1} & 0 & 0 & 0 \\
        \hline
    \end{tabular}
    \caption{Macierz blokująca $B_0$} \label{tab:b0d}
\end{table}
Dla macierzy blokującej przedstawionej w Tablicy~\ref{tab:b0d} znajdujemy jedno minimalne pokrycie kolumnowe
$\bm{L'=\{3,2,1,0\}}$ i wynikający z niego implikant prosty $\imp{d}=(0001)=x_3'x_2'x_1'x_0$.

\begin{table}[H]
    \centering
    \begin{tabular}[t]{ |c|c c c c| }
        \hline
        $k_1$ & 0 & 1 & 0 & 0 \\
        \hline\hline
        $B_1$ & $x_3$ & $x_2$ & $x_1$ & $x_0$ \\
        \hline
        & 0 & \textbf{1} & 0 & 0 \\
        & \sout{0} & \sout{1} & \sout{1} & \sout{0} \\
        & \sout{0} & \sout{1} & \sout{1} & \sout{1} \\
        & 0 & 0 & 0 & \textbf{1} \\
        & 0 & 0 & \textbf{1} & 0 \\
        & \sout{1} & \sout{1} & \sout{0} & \sout{0} \\
        & \sout{1} & \sout{1} & \sout{0} & \sout{1} \\
        \hline
    \end{tabular}
    \caption{Macierz blokująca $B_1$} \label{tab:b1d}
\end{table}
Dla macierzy blokującej przedstawionej w Tablicy~\ref{tab:b0d} znajdujemy jedno minimalne pokrycie kolumnowe
$\bm{L'=\{2,1,0\}}$ i wynikający z niego implikant prosty $\imp{d}=({*}100)=x_{2}x_1'x_0'$.

\begin{table}[H]
    \centering
    \begin{tabular}[t]{ |c|c c c c| }
        \hline
        $k_2$ & 0 & 1 & 1 & 1 \\
        \hline\hline
        $B_2$ & $x_3$ & $x_2$ & $x_1$ & $x_0$ \\
        \hline
        & \sout{0} & \sout{1} & \sout{1} & \sout{1} \\
        & \sout{0} & \sout{1} & \sout{0} & \sout{1} \\
        & 0 & \textbf{1} & 0 & 0 \\
        & 0 & 0 & \textbf{1} & 0 \\
        & 0 & 0 & 0 & \textbf{1} \\
        & \sout{1} & \sout{1} & \sout{1} & \sout{1} \\
        & \sout{1} & \sout{1} & \sout{1} & \sout{0} \\
        \hline
    \end{tabular}
    \caption{Macierz blokująca $B_2$} \label{tab:b2d}
\end{table}
Dla macierzy blokującej przedstawionej w Tablicy~\ref{tab:b2d} znajdujemy jedno minimalne pokrycie kolumnowe
$\bm{L'=\{2,1,0\}}$ i wynikający z niego implikant prosty $\imp{d}=({*}111)=x_{2}x_{1}x_{0}$.

\begin{table}[H]
    \centering
    \begin{tabular}[t]{ |c|c| }
        \hline
        $I_0$ & $0001$ \\
        $I_1$ & ${*}100$ \\
        $I_2$ & ${*}111$ \\
        \hline
    \end{tabular}
    \caption{Wszystkie implikanty proste} \label{tab:all-implicantsd}
\end{table}
%TODO: Dodać komentarz do tabeki wszytkich implikantów

\begin{table}[H]
    \centering
    \begin{tabular}[t]{ |c||c|c|c| }
        \hline
        & $I_0 = 0001$ & $I_1 = {*}100$ & $I_2 = {*}111$ \\
        \hline
        \hline
        $k_0 = 0001$ & \textbf{1} & 0 & 0 \\
        \hline
        $k_1 = 0100$ & 0 & \textbf{1} &  0 \\
        \hline
        $k_2 = 0111$ & 0 & 0 & \textbf{1} \\
        \hline
    \end{tabular}
    \caption{} \label{tab:min-blockd}
\end{table}
Minimalne pokrycie kolumnowe implikantów prostych $L' = \{I_0, I_1, I_2\}$.
$f_d(x_3, x_2, x_1, x_0) = x_3'x_2'x_1'x_0 + x_{2}x_1'x_0' + x_{2}x_{1}x_{0}$
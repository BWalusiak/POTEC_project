%! Suppress = MissingImport
\setcounter{implicant_counter}{0}

Funkcja $f_d$ rozpisana na macierze $F$ i $R$ ma postać:
\begin{center}
    \begin{tabular}[t]{ |c|c c c c| }
        \hline
        $F$ & $x_3$ & $x_2$ & $x_1$ & $x_0$ \\
        \hline
        $k_0$ & 0 & 0 & 0 & 0 \\
        $k_1$ & 0 & 0 & 1 & 0 \\
        $k_2$ & 0 & 0 & 1 & 1 \\
        $k_3$ & 0 & 1 & 0 & 1 \\
        $k_4$ & 0 & 1 & 1 & 0 \\
        $k_5$ & 1 & 0 & 0 & 0 \\
        $k_6$ & 1 & 0 & 0 & 1 \\
        \hline
    \end{tabular}
    \hspace{1cm}
    \begin{tabular}[t]{ |c|c c c c| }
        \hline
        $R$ & $x_3$ & $x_2$ & $x_1$ & $x_0$ \\
        \hline
        & 0 & 0 & 0 & 1 \\
        & 0 & 1 & 0 & 0 \\
        & 0 & 1 & 1 & 1 \\
        \hline
    \end{tabular}
\end{center}

Rozpoczynamy od wyznaczenia wszystkich implikantów prostych korzystając z macierzy blokujących dla kolejnych kostek
$k_0-k_6$.
\begin{table}[H]
    \centering
    \begin{tabular}[t]{ |c|c c c c| }
        \hline
        $k_0$ & 0 & 0 & 0 & 0 \\
        \hline\hline
        $B_0$ & $x_3$ & $x_2$ & $x_1$ & $x_0$ \\
        \hline
        & 0 & 0 & 0 & \textbf{1} \\
        & 0 & \textbf{1} & 0 & 0 \\
        & \sout{0} & \sout{1} & \sout{1} & \sout{1} \\
        \hline
    \end{tabular}
    \caption{Macierz blokująca $B_0$} \label{tab:b0d}
\end{table}
Dla macierzy blokującej przedstawionej w Tablicy~\ref{tab:b0d} znajdujemy jedno minimalne pokrycie kolumnowe
$\bm{L'=\{2,0\}}$ i wynikający z niego implikant prosty $\imp{d}=({*}0{*}0)=x_2'x_0'$.

\begin{table}[H]
    \centering
    \begin{tabular}[t]{ |c|c c c c| }
        \hline
        $k_1$ & 0 & 0 & 1 & 0 \\
        \hline\hline
        $B_1$ & $x_3$ & $x_2$ & $x_1$ & $x_0$ \\
        \hline
        & 0 & 0 & \textbf{1} & \textbf{1} \\
        & 0 & 1 & \textbf{1} & 0 \\
        & 0 & 1 & 0 & \textbf{1} \\
        \hline
    \end{tabular}
    \caption{Macierz blokująca $B_1$} \label{tab:b1d}
\end{table}
Dla macierzy blokującej przedstawionej w Tablicy~\ref{tab:b1d} znajdujemy trzy minimalne pokrycia kolumnowe
$\bm{L'=\{1,0\}}$, $L'=\{2,0\}$, $L'=\{2,1\}$ i
wynikające z nich implikanty proste $\imp{d}=({*}{*}10)=x_{1}x_0'$, $\imp{d}=({*}0{*}0)=x_2'x_0'$, $\imp{d}=({*}01{*})=x_2'x_1$.

\begin{table}[H]
    \centering
    \begin{tabular}[t]{ |c|c c c c| }
        \hline
        $k_2$ & 0 & 0 & 1 & 1 \\
        \hline\hline
        $B_2$ & $x_3$ & $x_2$ & $x_1$ & $x_0$ \\
        \hline
        & 0 & 0 & \textbf{1} & 0 \\
        & \sout{0} & \sout{\textbf{1}} & \sout{\textbf{1}} & \sout{1} \\
        & 0 & \textbf{1} & 0 & 0 \\
        \hline
    \end{tabular}
    \caption{Macierz blokująca $B_2$} \label{tab:b2d}
\end{table}
Dla macierzy blokującej przedstawionej w Tablicy~\ref{tab:b2d} znajdujemy jedno minimalne pokrycie kolumnowe
$\bm{L'=\{2,1\}}$ i wynikający z niego implikant prosty $\imp{d}=({*}01{*})=x_2'x_1$.

\begin{table}[H]
    \centering
    \begin{tabular}[t]{ |c|c c c c| }
        \hline
        $k_3$ & 0 & 1 & 0 & 1 \\
        \hline\hline
        $B_3$ & $x_3$ & $x_2$ & $x_1$ & $x_0$ \\
        \hline
        & 0 & \textbf{1} & 0 & 0 \\
        & 0 & 0 & 0 & \textbf{1} \\
        & 0 & 0 & \textbf{1} & 0 \\
        \hline
    \end{tabular}
    \caption{Macierz blokująca $B_3$} \label{tab:b3d}
\end{table}
Dla macierzy blokującej przedstawionej w Tablicy~\ref{tab:b3d} znajdujemy jedno minimalne pokrycie kolumnowe
$\bm{L'=\{2,1,0\}}$ i wynikający z niego implikant prosty $\imp{d}=({*}101)=x_{2}x_1'x_0$.

\begin{table}[H]
    \centering
    \begin{tabular}[t]{ |c|c c c c| }
        \hline
        $k_4$ & 0 & 1 & 1 & 0 \\
        \hline\hline
        $B_4$ & $x_3$ & $x_2$ & $x_1$ & $x_0$ \\
        \hline
        & \sout{0} & \sout{1} & \sout{\textbf{1}} & \sout{\textbf{1}} \\
        & 0 & 0 & \textbf{1} & 0 \\
        & 0 & 0 & 0 & \textbf{1} \\
        \hline
    \end{tabular}
    \caption{Macierz blokująca $B_4$} \label{tab:b4d}
\end{table}
Dla macierzy blokującej przedstawionej w Tablicy~\ref{tab:b4d} znajdujemy jedno minimalne pokrycie kolumnowe
$\bm{L'=\{1,0\}}$ i wynikający z niego implikant prosty $\imp{d}=({*}{*}10)=x_{1}x_0'$.

\begin{table}[H]
    \centering
    \begin{tabular}[t]{ |c|c c c c| }
        \hline
        $k_5$ & 1 & 0 & 0 & 0 \\
        \hline\hline
        $B_5$ & $x_3$ & $x_2$ & $x_1$ & $x_0$ \\
        \hline
        & \textbf{1} & 0 & 0 & 1 \\
        & \textbf{1} & 1 & 0 & 0 \\
        & \sout{\textbf{1}} & \sout{1} & \sout{1} & \sout{1} \\
        \hline
    \end{tabular}
    \caption{Macierz blokująca $B_5$} \label{tab:b5d}
\end{table}
Dla macierzy blokującej przedstawionej w Tablicy~\ref{tab:b5d} znajdujemy jedno minimalne pokrycie kolumnowe
$\bm{L'=\{3\}}$ i wynikający z niego implikant prosty $\imp{d}=(1{*}{*}{*})=x_3$.

\begin{table}[H]
    \centering
    \begin{tabular}[t]{ |c|c c c c| }
        \hline
        $k_6$ & 1 & 0 & 0 & 1 \\
        \hline\hline
        $B_6$ & $x_3$ & $x_2$ & $x_1$ & $x_0$ \\
        \hline
        & \textbf{1} & 0 & 0 & 0 \\
        & \sout{\textbf{1}} & \sout{1} & \sout{0} & \sout{1} \\
        & \sout{\textbf{1}} & \sout{1} & \sout{1} & \sout{0} \\
        \hline
    \end{tabular}
    \caption{Macierz blokująca $B_6$} \label{tab:b6d}
\end{table}
Dla macierzy blokującej przedstawionej w Tablicy~\ref{tab:b6d} znajdujemy jedno minimalne pokrycie kolumnowe
$\bm{L'=\{3\}}$ i wynikający z niego implikant prosty $\imp{d}=(1{*}{*}{*})=x_3$.

\begin{table}[H]
    \centering
    \begin{tabular}[t]{ |c|c| }
        \hline
        $I_0$ & ${*}0{*}0$ \\
        $I_1$ & ${*}{*}10$ \\
        \sout{$I_2$} & \sout{${*}0{*}0$} \\
        $I_3$ & ${*}01{*}$ \\
        \sout{$I_4$} & \sout{${*}01{*}$} \\
        $I_5$ & ${*}101$ \\
        \sout{$I_6$} & \sout{${*}{*}10$} \\
        $I_7$ & $1{*}{*}{*}$ \\
        \sout{$I_8$} & \sout{$1{*}{*}{*}$} \\
        \hline
    \end{tabular}
    \caption{Wszystkie implikanty proste} \label{tab:all-implicantsd}
\end{table}
%TODO: Dodać komentarz do tabeki wszytkich implikantów

\begin{table}[H]
    \centering
    \begin{tabular}[t]{ |c||c|c|c|c|c| }
        \hline
        & $I_0 = {*}0{*}0$ & $I_1 = {*}{*}10$ & $I_3 = {*}01{*}$ & $I_5 = {*}101$ & $I_7 = 1{*}{*}{*}$ \\
        \hline
        \hline
        $k_0 = 0000$ & \textbf{1} & 0 & 0 & 0 & 0 \\
        \hline
        \sout{$k_1 = 0010$} &  \sout{\textbf{1}} &  \sout{\textbf{1}} &  \sout{\textbf{1}} & \sout{0} & \sout{0} \\
        \hline
        $k_2 = 0011$ & 0 & 0 & \textbf{1} & 0 & 0 \\
        \hline
        $k_3 = 0101$ & 0 & 0 & 0 & \textbf{1} & 0 \\
        \hline
        $k_4 = 0110$ & 0 & \textbf{1} & 0 & 0 & 0 \\
        \hline
        \sout{$k_5 = 1000$} &  \sout{\textbf{1}} & \sout{0} & \sout{0} & \sout{0} &  \sout{\textbf{1}} \\
        \hline
        $k_6 = 1001$ & 0 & 0 & 0 & 0 & \textbf{1} \\
        \hline
    \end{tabular}
    \caption{} \label{tab:min-blockd}
\end{table}
Minimalne pokrycie kolumnowe implikantów prostych $L' = \{I_0, I_1, I_3, I_5, I_7\}$.
$f_d(x_3, x_2, x_1, x_0) = x_2'x_0' + x_{1}x_0' + x_2'x_1 + x_3$
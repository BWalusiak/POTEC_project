%! Suppress = MissingImport
\setcounter{implicant_counter}{0}

Funkcja $f_c$ rozpisana na macierze $F$ i $R$ ma postać:
\begin{center}
    \begin{tabular}[t]{ |c|c c c c|}
        \hline
        $F$ & $x_3$ & $x_2$ & $x_1$ & $x_0$ \\
        \hline
        $k_0$ & 0 & 0 & 0 & 0 \\
        $k_1$ & 0 & 0 & 0 & 1 \\
        $k_2$ & 0 & 0 & 1 & 1 \\
        $k_3$ & 0 & 1 & 0 & 0 \\
        $k_4$ & 0 & 1 & 0 & 1 \\
        $k_5$ & 0 & 1 & 1 & 0 \\
        $k_6$ & 0 & 1 & 1 & 1 \\
        $k_7$ & 1 & 0 & 0 & 0 \\
        $k_8$ & 1 & 0 & 0 & 1 \\
        \hline
    \end{tabular}
    \hspace{1cm}
    \begin{tabular}[t]{ |c|c c c c| }
        \hline
        $R$ & $x_3$ & $x_2$ & $x_1$ & $x_0$ \\
        \hline
        & 0 & 0 & 1 & 0 \\
        \hline
    \end{tabular}
\end{center}

Rozpoczynamy od wyznaczenia wszystkich implikantów prostych korzystając z macierzy blokujących dla kolejnych kostek
$k_0-k_8$.
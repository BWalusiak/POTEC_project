%! Suppress = MissingImport
\setcounter{implicant_counter}{0}

Funkcja $f_c$ rozpisana na macierze $F$ i $R$ ma postać:
\begin{center}
    \begin{tabular}[t]{ |c|c c c c|}
        \hline
        $F$ & $x_3$ & $x_2$ & $x_1$ & $x_0$ \\
        \hline
        $k_0$ & 0 & 0 & 1 & 0 \\
        \hline
    \end{tabular}
    \hspace{1cm}
    \begin{tabular}[t]{ |c|c c c c| }
        \hline
        $R$ & $x_3$ & $x_2$ & $x_1$ & $x_0$ \\
        \hline
        & 0 & 0 & 0 & 0 \\
        & 0 & 0 & 0 & 1 \\
        & 0 & 0 & 1 & 1 \\
        & 0 & 1 & 0 & 0 \\
        & 0 & 1 & 0 & 1 \\
        & 0 & 1 & 1 & 0 \\
        & 0 & 1 & 1 & 1 \\
        & 1 & 0 & 0 & 0 \\
        & 1 & 0 & 0 & 1 \\
        \hline
    \end{tabular}
\end{center}

Rozpoczynamy od wyznaczenia wszystkich implikantów prostych korzystając z macierzy blokujących dla kostki
$k_0$.

\begin{table}[H]
    \centering
    \begin{tabular}[t]{ |c|c c c c| }
        \hline
        $k_0$ & 0 & 0 & 1 & 0 \\
        \hline\hline
        $B_0$ & $x_3$ & $x_2$ & $x_1$ & $x_0$ \\
        \hline
        & 0 & 0 & \textbf{1} & 0 \\
        & \sout{0} & \sout{0} & \sout{1} & \sout{1} \\
        & 0 & 0 & 0 & \textbf{1} \\
        & \sout{0} & \sout{1} & \sout{1} & \sout{0} \\
        & \sout{0} & \sout{1} & \sout{1} & \sout{1} \\
        & 0 & \textbf{1} & 0 & 0 \\
        & \sout{0} & \sout{1} & \sout{0} & \sout{1} \\
        & \sout{1} & \sout{0} & \sout{1} & \sout{0} \\
        & \sout{1} & \sout{0} & \sout{1} & \sout{1} \\
        \hline
    \end{tabular}
    \caption{Macierz blokująca $B_0$} \label{tab:b0c}
\end{table}

Dla macierzy blokującej przedstawionej w Tablicy~\ref{tab:b0c} znajdujemy jedno minimalne pokrycie kolumnowe
$\bm{L'=\{2,1,0\}}$ i wynikający z niego implikant prosty $\imp{c}=({*}010)=x_0'x_{1}x_2'$.

Minimalne pokrycie kolumnowe implikantów prostych $L' = \{I_0\}$.\newline
$f_d(x_3, x_2, x_1, x_0) = x_0'x_{1}x_2'$